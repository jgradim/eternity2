% This is LLNCS.DEM the demonstration file of
% the LaTeX macro package from Springer-Verlag
% for Lecture Notes in Computer Science,
% version 2.4 for LaTeX2e as of 16. April 2010
%
\documentclass{llncs}
%
\usepackage{makeidx}  % allows for indexgeneration
\usepackage[utf8]{inputenc}
%
\begin{document}
%
\frontmatter          % for the preliminaries
%
\pagestyle{headings}  % switches on printing of running heads
\addtocmark{Eternity II} % additional mark in the TOC
%

\mainmatter              % start of the contributions
%
\title{Eternity II}
%
\titlerunning{Eternity II}  % abbreviated title (for running head)
%                                     also used for the TOC unless
%                                     \toctitle is used
%
\author{João Gradim \and Mário Carneiro}
%
\authorrunning{João Gradim\inst{1} et Mário Carneiro\inst{1}} % abbreviated author list (for running head)
%
%%%% list of authors for the TOC (use if author list has to be modified)
\tocauthor{João Gradim and Mário Carneiro}
%
\institute{Faculdade de Engenharia da Universidade do Porto,\\
\email{ei05030@fe.up.pt}, \email{ei04051@fe.up.pt}}

\maketitle              % typeset the title of the contribution

\begin{abstract}
\end{abstract}
%
\section{Introduction}

This section will describe why we are picking up on the work previously developed by Fábio Aguiar and Sara Carvalho, and what methodologies we will try

\section{Problem Description}

\begin{itemize}
	\item Eternity II puzzle description
	\item How the diamonds approach is different from a standard, to-piece approach, and why it may yield better results
\end{itemize}

\section{New approaches}

Methodologies and approaches taken when trying new methods to improve solving times of simplified Eternity II puzzles

\subsection{Eternity II Editor}

\subsection{Swap methods}

\subsection{Iterative placement patterns}

\section{Results}

Experimental results obtained from the application of the new methodologies to the diamonds board approach

\section{Conclusions}

%
% ---- Bibliography ----
%
\bibliography{mpes}
\end{document}
